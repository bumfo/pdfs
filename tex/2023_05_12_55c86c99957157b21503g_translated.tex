\documentclass[10pt]{article} 
\usepackage[utf8]{inputenc} 
\usepackage[T1]{fontenc} 
\usepackage{amsmath} 
\usepackage{amsfonts} 
\usepackage{amssymb} 
\usepackage[version=4]{mhchem} 
\usepackage{stmaryrd} 
\usepackage{mathrsfs} 

\title{Brownian Motion of Spins } 


\author{Ryogo Kubo and Natsuki HasHitsume*\\
Department of Physics, University of Tokyo, Tokyo\\
*Department of Physics, Ochanomizu University\\
Bunkyo-ku, Tokyo} 
\date{} 


\begin{document} 
\maketitle 
(1970年11月4日收到)

\begin{abstract} 
讨论了自旋的布朗运动的随机模型。如果假定摩擦阻力伴随着引起布朗运动的随机场,自旋运动的随机方程就会导致福克-普朗克方程,从而保证接近热平衡。假设有Landau-Lifshitz类型的摩擦,平均磁矩被证明是服从Bloch方程的。
\end{abstract} 

\section{§1. Introduction} 
在一个凝聚系统中,自旋的运动由于与周围环境相互作用的影响而变得非常复杂。然而,这种复杂性本身允许一个理想化,即引入一个随机的模型。在这里,我们认为最简单的理想化是布朗运动模型,其中自旋与周围环境的相互作用由一个随机波动的磁场来表示。那么自旋的运动将由以下类型的随机方程来描述

\[
\frac{d \boldsymbol{M}}{d t}=\gamma\left\{\boldsymbol{H}(t)+\boldsymbol{H}^{\prime}(t)\right\} \times \boldsymbol{M}
\] 

其中\(\boldsymbol{H}(t)\) 是观察过程中控制的外部场,\(\boldsymbol{H}^{\prime}(t)\) 是随机磁场。几年前,一位作者在他的随机Liouville方程理论的基础上简要地讨论了这样一个经典自旋模型。\({ }^{2}\) Yoshimori和Korringa\({ }^{3)}\) 在量子自旋的随机海森堡方程的基础上发展了一个类似的理论。当波动场被假定为高斯过程并满足缩小条件时,\({ }^{1)}\) 这些理论给出了磁化的Bloch方程。然而,这样得出的方程有一个缺陷。也就是说,磁化并没有向其对应于给定外场的平衡值放松,而是向零值放松。我们不能在理论中加入有限的温度和相应的平衡磁化。

为了克服这一困难,已经做了一些尝试。Korringa \({ }^{4)}\) 在随机薛定谔方程中引入了一个复杂的时间。它的虚部对应于倒数的温度。使用这个形式主义,Korringa等人\({ }^{5)}\) 得到了一个修改的Bloch方程,通过它,磁化放松到一个有限的平衡值。Nakano和Yoshimori\({ }^{6)}\) 借助于Kikuchi开发的路径概率方法,提出了\(^{2}\) 另一种方法。"从本质上讲,他们假设了一个查普曼-斯莫楚夫斯基方程(一个主方程)的类型

\[
\frac{d p_{i}}{d t}=\sum_{j} \theta_{i j}\left(p_{j} e^{\left(E_{j}-E_{i}\right) / 2 k T}-p_{i} e^{\left(E_{i}-E_{j}\right) / 2 k T}\right)
\] 

其中\(i\) 和\(j\) 指的是所考虑的系统(自旋)的状态,\(p_{j}\) 是在\(j\) 的状态下发现系统的概率,\(E_{j}\) 是\(j\) 的状态的能量。他们使用(1)型的随机运动方程来确定过渡概率\(\theta_{i j}\) 。然而,他们引入了相当任意的因素\(\exp \left( \pm\left(E_{i}-E_{j}\right) / 2 k T\right)\) ,以确保在有限的温度下接近平衡。

在这里,我们从一个有点不同的观点来考虑这个问题,它对应于布朗运动的朗温理论。众所周知,在布朗运动理论中,作用在布朗粒子上的随机力必然与摩擦力相结合,这一事实代表了一个非常普遍的自然规律--波动耗散定理。\({ }^{8)}\) 这意味着随机运动方程,即公式(1),必须辅以摩擦力。那么,公式(1)将被这样一个方程所取代,即

\[
\frac{d \boldsymbol{M}}{d t}=\gamma\left\{\boldsymbol{H}(t)+\boldsymbol{H}^{\prime}(t)-\kappa \frac{d \boldsymbol{M}}{d t}\right\} \times \boldsymbol{M}
\] 

摩擦力以某种方式与随机场\(\boldsymbol{H}^{\prime}\) 相关,由波动耗散定理决定。

与我们熟悉的自由粒子或谐波振荡器的布朗运动理论相比,自旋的布朗运动涉及一些复杂的问题。这些复杂性来自于公式(1)或(2)的准非线性结构。磁化\(\boldsymbol{M}(t)\) 在随机场\(\boldsymbol{H}^{\prime}(t)\) 中不是线性的,因此,不能采用简单的谐波分析。事实上,在一般情况下,即使假定基本过程\(\boldsymbol{H}^{\prime}(t)\) 是简单的,例如高斯过程,公式(1)或(2)都很难解决。那么\(\boldsymbol{M}(t)\) 的随机过程就相当复杂,不容易进行分析处理。对于这样的问题,已经进行了一些尝试\(^{9) 10)}\) ,目的是将其应用于低场的磁共振。

然而,在本文中,我们不会去研究这样一个复杂的问题,但我们只限于缩小极限的情况,其中随机场是弱的,波动非常快,所以条件是

\[
\Delta \tau_{c} \ll 1, \quad \Delta \simeq \gamma H^{\prime}
\] 

得到满足,其中\(H^{\prime}\) 是平均振幅,而\(\tau_{f}\) 是随机场的相关时间。在这个极限中,\(\boldsymbol{M}(t)\) 的随机过程是由福克-普朗克方程描述的。在平均情况下,磁化矢量将被证明服从布洛赫方程,松弛到有限的热平衡。\({ }^{11)}\) 

\section{§2. A classical spin rotating without friction} 
为了使问题的性质清楚,我们在这里从最简单的波动自旋运动的随机模型开始,如公式(1)所代表的。对于处理这种类型的随机过程,最方便的是引入随机的Liouville方程\({ }^{2)}\) ,在这种情况下写为

\[
\frac{\partial}{\partial t} \rho(\boldsymbol{M}, t)=-\frac{\partial}{\partial \boldsymbol{M}} \cdot\left(\dot{\boldsymbol{M}}_{\rho}(\boldsymbol{M}, t)\right)
\] 

或

\[
\frac{\partial}{\partial t} \rho=-i \mathcal{L}(t) \rho
\] 

其中Liouville算子\(\mathcal{L}\) 被写为

\[
\mathcal{L}=\mathcal{L}_{0}+\mathcal{L}^{\prime}(t)
\] 

其中

\[
\begin{aligned}
& \mathcal{L}_{0}=\omega_{0} L_{z}, \\
& \mathcal{L}^{\prime}(t)=\gamma \boldsymbol{H}^{\prime}(t) \cdot \boldsymbol{L}, \\
& \boldsymbol{L}=-i \boldsymbol{M} \times \frac{\partial}{\partial \boldsymbol{M}} .
\end{aligned}
\] 

\(\boldsymbol{H}_{0}\) 的方向被当作\(z\) 轴,\(\omega_{0}=\gamma H_{0}\) 是\(\boldsymbol{H}_{0}\) 中自旋的拉摩尔频率。算子\(\boldsymbol{L}\) 是\(\boldsymbol{M}\) 空间中的角动量算子。方程(4)或(5)是概率密度\(\rho(\boldsymbol{M}, t)\) 的随机Liouville方程,用于寻找\(t\) 时间的矢量\(\boldsymbol{M}\) ,在\(\boldsymbol{M}\) 空间的给定点的邻近区域。它包含了随机过程\(\boldsymbol{H}^{\prime}(t)\) ,从中应确定\(\boldsymbol{M}(t)\) 的随机特性。对于初始分布\(\rho(\boldsymbol{M}, 0)\) 和过程\(\boldsymbol{H}^{\prime}(t)\) 的给定样本,方程(5)可以正式求解为

\[
\rho(\boldsymbol{M}, t)=\exp _{\leftarrow}\left\{-i \int_{0}^{t} \mathcal{L}\left(t^{\prime}\right) d t^{\prime}\right\} \rho(\boldsymbol{M}, 0)
\] 

具体来说,对于

\[
\rho(\boldsymbol{M}, 0)=\delta\left(\boldsymbol{M}-\boldsymbol{M}^{\prime}\right)
\] 

的最终分布可以写成

\[
\rho(\boldsymbol{M}, t)=\left(\boldsymbol{M}\left|\exp _{\leftarrow}\left\{-i \int_{0}^{t} \mathcal{L}\left(t^{\prime}\right) d t^{\prime}\right\}\right| \boldsymbol{M}^{\prime}\right)
\] 

其中括号符号用于表示变换的积分核。这个变换在\(\boldsymbol{H}^{\prime}(t)\) 的整个集合上被平均化,以得到\(\boldsymbol{M}^{\prime}\) 到\(\boldsymbol{M}\) 在时间间隔\((0, t)\) 的过渡概率

\[
f\left(\boldsymbol{M} t \mid \boldsymbol{M}^{\prime} 0\right)=\left(\boldsymbol{M}\left|\left\langle\exp _{\leftarrow}\left\{-i \int_{0}^{t} \mathcal{L}\left(t^{\prime}\right) d t^{\prime}\right\}\right\rangle\right| \boldsymbol{M}^{\prime}\right)
\] 

其中\textbackslash  langle\textbackslash  rangle指的是集合平均。因此,我们可以称

\[
\Phi(t)=\left\langle\exp _{\leftarrow}\left\{-i \int_{0}^{t} \mathcal{L}\left(t^{\prime}\right) d t^{\prime}\right\}\right\rangle
\] 

是过程\(\boldsymbol{M}(t)\) 的过渡算子。现在的问题是计算这个过渡算子(13)并找到相关物理量的平均数。

这不是一项容易的任务。一般来说,除非引入一些简化,否则不可能推导出过渡概率的分析性表达。我们可以合理地假设过程\(\boldsymbol{H}^{\prime}(t)\) 是高斯的。Toyabe \(^{10)}\) 和Toyabe和Kubo.已经更详细地处理了这种假设下的布朗自旋运动。\({ }^{9)}\) 即使是这样的简化,也不可能进行精确的处理,因为不同时间点的算子\(\mathcal{L}(t)\) 是不能互换的。只有在缩小条件(3)的限制下,理论才变得简单。然而,如果满足收窄条件(3),不仅对高斯调制\(\boldsymbol{H}^{\prime}(t)\) ,而且对更普遍的一类过程\(\boldsymbol{H}^{\prime}(t)\) 也可以实现同样的简化,只要它足够适度。我们在这里并不试图给出温和过程的严格定义,但我们指的是某种中心极限定理对其成立的过程。因此,一个由一系列不规则脉冲组成的过程被排除在外。

在由以下定义的相互作用表示中

\[
\hat{\rho}=e^{i \mathcal{L}_{0} t_{\rho}}
\] 

和

\[
\Omega(t)=e^{i \mathcal{L}_{0} t} \mathcal{L}^{\prime}(t) e^{-i \mathcal{L}_{0} t}
\] 

过渡算子(13)被写成

\[
\Phi(t)=e^{-i \mathcal{L}_{0} t}\left\langle\exp _{\leftarrow}\left(-i \int_{0}^{t} \Omega\left(t^{\prime}\right) d t^{\prime}\right)\right\rangle e^{i \mathcal{L}_{0} t}
\] 

其中,只有算子\(\Omega(t)\) 包含随机场\(\boldsymbol{H}^{\prime}(t)\) ,对于该场,我们假定

\[
\begin{aligned}
& \left\langle\boldsymbol{H}^{\prime}(t)\right\rangle=0, \\
& \left\langle H_{\alpha}^{\prime}\left(t_{1}\right) H_{\beta}^{\prime}\left(t_{2}\right)\right\rangle=0, \quad \alpha \neq \beta, \\
& \gamma^{2}\left\langle H_{x}^{\prime}\left(t_{1}\right) H_{x}^{\prime}\left(t_{2}\right)\right\rangle=\gamma^{2}\left\langle H_{y}^{\prime}\left(t_{1}\right) H_{y}^{\prime}\left(t_{2}\right)\right\rangle=\psi_{\perp}\left(t_{1}-t_{2}\right),
\end{aligned}
\] 

\[
\gamma^{2}\left\langle H_{z}^{\prime}\left(t_{1}\right) H_{z}^{\prime}\left(t_{2}\right)\right\rangle=\psi_{\|}\left(t_{1}-t_{2}\right)
\] 

当公式(15)右侧的有序指数算子被展开时,第一项变成了

\[
\begin{aligned}
& -\int_{0}^{t} d t_{1} \int_{0}^{t_{1}} d t_{2} e^{i \mathcal{L}_{0} t_{1}}\left\langle\mathcal{L}^{\prime}\left(t_{1}\right) e^{-i \mathcal{L}_{0}\left(t_{1}-t_{2}\right)} \mathcal{L}^{\prime}\left(t_{2}\right) e^{i \mathcal{L}_{0}\left(t_{1}-t_{2}\right)}\right\rangle e^{-i \mathcal{L}_{0} t_{1}} \\
\simeq & -\int_{0}^{t} d t_{1} e^{i \mathcal{L}_{0} t_{1}} \int_{0}^{\infty}\left\langle\mathcal{L}^{\prime}\left(t_{1}\right) e^{-i \mathcal{L}_{0} \tau} \mathcal{L}^{\prime}\left(t_{1}-\tau\right) e^{i \mathcal{L}_{0} \tau}\right\rangle e^{-i \mathcal{L}_{0} t_{1}} \\
= & -\int_{0}^{t} d t_{1} e^{i \mathcal{L}_{0} t_{1}}\left\{i \delta \omega_{0} L_{z}+\frac{1}{2 \tau_{1}}\left(L_{x}^{2}+L_{y}^{2}\right)+\frac{1}{2 \tau_{0}} L_{z}^{2}\right\} e^{-i \mathcal{L}_{0} t_{1}},
\end{aligned}
\] 

其中

\[
\begin{aligned}
& \frac{1}{2 \tau_{0}}=\int_{0}^{\infty} \psi_{/ /}(t) d t, \\
& \frac{1}{2 \tau_{1}}=\int_{0}^{\infty} \psi_{\perp}(t) \cos \omega_{0} t d t, \\
& \delta \omega_{0}=\int_{0}^{\infty} \psi_{\perp}(t) \sin \omega_{0} t d t .
\end{aligned}
\] 

在转换公式(19)时,我们使用了假设,即\(\psi_{\|}\)or \(\psi_{\perp}\) 的相关时间\(\tau_{c}\) 与所考虑的时间\(t\) 相比非常短,即

\[
t \gg \tau_{c}
\] 

因此允许将\(t_{2}\) 的下限扩展到\(-\infty\) 。式(15)中平均有序指数算子的高阶扩展项可以用类似的方式进行评估。只要随机场\(\boldsymbol{H}^{\prime}(t)\) 是适度的,并且满足缩小条件(3),只有那些重要的项被简化为算子(19)的乘积,因此过渡算子(15)的形式为

\[
\begin{aligned}
\Phi(t)= & e^{-i \mathcal{L}_{0} t} \exp \left[-\int_{0}^{t} d t_{1} e^{i \mathcal{L}_{0} t}\left\{i \delta \omega_{0} L_{z}+\frac{1}{2 \tau_{1}}\left(L_{x}^{2}+L_{y}^{2}\right)\right.\right. \\
& \left.\left.+\frac{1}{2 \tau_{0}} L_{z}^{2}\right\} e^{-i \mathcal{L}_{0} t}\right] e^{i \mathcal{L}_{0} t} .
\end{aligned}
\] 

通过对上述表达式与\(t\) 进行微分,我们发现

\[
\frac{\partial}{\partial t} \Phi(t)=-i\left(\omega_{0}+\delta \omega_{0}\right) L_{z}+\frac{1}{2 \tau_{1}}\left(L_{x}^{2}+L_{y}^{2}\right)+\frac{1}{2 \tau_{0}} L_{z}^{2}
\] 

因此,过渡概率\(f\) 是该方程的基本解、

\[
\left[\frac{\partial}{\partial t}+i\left(\omega_{0}+\delta \omega_{0}\right) L_{z}+\frac{1}{2 \tau_{1}}\left(L_{x}^{2}+L_{y}^{2}\right)+\frac{1}{2 \tau_{0}} L_{z}^{2}\right] f=0 .
\] 

这可以写成

\[
\left[\frac{\partial}{\partial t}+i \gamma(1+\delta) \boldsymbol{H}_{0} \cdot \mathbf{L}-(i \boldsymbol{L} \cdot \boldsymbol{D} \cdot i \boldsymbol{L})\right] f=0
\] 

其中\(\boldsymbol{H}_{0}\) 是外场,扩散张量\(\boldsymbol{D}\) 定义为

\[
\boldsymbol{D}=D_{1}\left(1-\frac{\boldsymbol{H}_{0} \boldsymbol{H}_{0}}{H_{0}^{2}}\right)+D_{0} \frac{\boldsymbol{H}_{0} \boldsymbol{H}_{0}}{H_{0}^{2}}, \quad D_{1}=\frac{1}{2 \tau_{1}}, \quad D_{0}=\frac{1}{2 \tau_{0}}
\] 

松弛时间\(\tau_{0}\) 和\(\tau_{1}\) 以及分数位移\(\delta=\delta \omega_{0} / \omega_{0}\) 由公式(20)定义。方程(24)可用于随时间变化的外场\(\boldsymbol{H}_{0}(t)\) ,只要其时间变化在相关时间\(\tau_{c}\) 中是缓慢的。

磁矩的平均值

\[
\langle\boldsymbol{M}(t)\rangle=\int \boldsymbol{M} f(\boldsymbol{M}, t) d \boldsymbol{M}
\] 

的平均值很容易被发现,以满足布洛赫方程

\[
\begin{aligned}
& \frac{d}{d t}\left\langle M_{x}\right\rangle=-\omega_{0}^{\prime}\left\langle M_{y}\right\rangle-\frac{\left\langle M_{x}\right\rangle}{T_{2}}, \quad \omega_{0}^{\prime}=\omega_{0}+\delta \omega_{0}, \\
& \frac{d}{d t}\left\langle M_{y}\right\rangle=\omega_{0}^{\prime}\left\langle M_{x}\right\rangle-\frac{\left\langle M_{y}\right\rangle}{T_{2}}, \\
& \frac{d\left\langle M_{z}\right\rangle}{d t}=\frac{\left\langle M_{z}\right\rangle}{T_{1}},
\end{aligned}
\] 

其中纵向和横向松弛时间\(T_{1}\) 和\(T_{2}\) 由以下公式给出

\[
T_{1}=\tau_{1}, \quad \frac{1}{T_{2}}=\frac{1}{2}\left(\frac{1}{\tau_{0}}+\frac{1}{\tau_{1}}\right)
\] 

根据公式(26),磁矩放松到零。正如在介绍中提到的,这必然是由缺少摩擦项的公式(1)所引起的。

\section{§3. A classical spin rotating with friction} 
我们现在考虑一个经典自旋,它在旋转时受到摩擦阻力。按照朗道和利夫希茨\(^{12)}\) ,我们可以假设运动方程、

\[
\frac{d}{d t} \boldsymbol{M}=\gamma \boldsymbol{H} \times \boldsymbol{M}-\eta(\boldsymbol{H} \times \boldsymbol{M}) \times \boldsymbol{M} \text {. }
\] 

这是从\(d \boldsymbol{M} / d t\) 是\(\boldsymbol{H} \times \boldsymbol{M}\) 和\((\boldsymbol{H} \times \boldsymbol{M}) \times \boldsymbol{M}\) 的线性组合的观察中得到的,如果\(\boldsymbol{M}\) 的大小应被保留。我们同样可以假设另一种类型的方程、

\[
\frac{d \boldsymbol{M}}{d t}=\gamma^{\prime}\left(\boldsymbol{H}-\kappa \frac{d \boldsymbol{M}}{d t}\right) \times \boldsymbol{M}
\] 

正如在介绍中提到的那样。后者的方程可以被转化为

\[
\frac{d \boldsymbol{M}}{d t}=\frac{\gamma^{\prime}}{1+\left(\gamma^{\prime} \kappa\right)^{2} M^{2}}(\boldsymbol{H}-\kappa \boldsymbol{H} \times \boldsymbol{M}) \times \boldsymbol{M}
\] 

这与公式(28)相同,只是有重正化系数。因此我们在这里使用公式(28)。

如果场\(\boldsymbol{H}\) 由外部场\(\boldsymbol{H}_{0}(t)\) 和随机场\(\boldsymbol{H}^{\prime}(t)\) 组成,即

\[
\boldsymbol{H}(t)=\boldsymbol{H}_{0}(t)+\boldsymbol{H}^{\prime}(t)
\] 

式(28)是我们对经典自旋的随机运动方程。这可以进一步简化为

\[
\frac{d}{d t} \boldsymbol{M}(t)=\gamma\left(\boldsymbol{H}_{0}(t)+\boldsymbol{H}^{\prime}(t)\right) \times \boldsymbol{M}-\eta\left(\boldsymbol{H}_{0} \times \boldsymbol{M}\right) \times \boldsymbol{M}
\] 

通过省略摩擦项中的随机场。正如我们将在后面看到的,爱因斯坦关系要求

\[
\eta=1 / 2 \tau_{1} k T
\] 

所以我们有数量级的估计、

\[
O\left(n \boldsymbol{H}^{\prime} \times \boldsymbol{M}\right)=H^{\prime} M / 2 \tau_{1} k T=\Delta^{2} \tau_{c} H^{\prime} M / k T
\] 

由于缩小的条件,也由于通常的实验条件,即

\[
H^{\prime} M \ll k T
\] 

摩擦项的随机部分的影响要比随机场的直接影响小得多,因此,上述简化是合理的。

在粗粒化(21)、窄化(3)和\(\boldsymbol{H}^{\prime}(t)\) 的适度行为的假设下,从(31)推导出Fokker-Planck方程是直接的。未扰动的Liouville算子,公式(7),现在被替换为

\[
\mathcal{L}_{0}(t)=\omega_{0} L_{z}+\eta H_{0}\left(L_{x} M_{y}-L_{y} M_{x}\right)
\] 

公式(33)右侧第二项的存在似乎会引入复杂的问题。然而,与(32)相同的估计表明,与随机场的相关时间\(\tau_{c}\) 所表征的时间速率相比,它只引起\(\exp \left(i \mathcal{L}_{0} t\right)\) 的非常缓慢的时间变化,因此,公式(19)中括号内的算子不需要改变。因此,公式(24)的唯一变化是增加了术语

\[
\frac{\partial}{\partial \boldsymbol{M}} \eta\left(\boldsymbol{H}_{0} \times \boldsymbol{M}\right) \times \boldsymbol{M} f=i \eta \boldsymbol{L}\left(\boldsymbol{M} \times \boldsymbol{H}_{0}\right) f
\] 

所以公式(24)现在被概括为

\[
\left[\frac{\partial}{\partial t}+i \gamma(1+\boldsymbol{\delta}) \boldsymbol{H}_{0} \boldsymbol{L}+i \boldsymbol{L} \boldsymbol{D}\left(i \boldsymbol{L}-\frac{\boldsymbol{M} \times \boldsymbol{H}_{0}}{k T}\right)\right] f=0
\] 

其中包括与随机场相关的摩擦效应。这里我们已经使用了(31)的关系。在由以下定义的极坐标中

\[
M_{x}=M \sin \theta \cos \phi, \quad M_{y}=M \sin \theta \sin \phi, \quad M_{z}=M \cos \theta
\] 

公式(35)被写成

\[
\begin{aligned}
\left(\frac{\partial}{\partial t}+\right. & \left.\omega_{0}^{\prime} \frac{\partial}{\partial \phi}\right) f=\left[\frac{1}{\sin \theta} \frac{\partial}{\partial \theta}\left\{\sin \theta D_{1}\left(\frac{\partial}{\partial \theta}+\frac{H_{0} M \sin \theta}{k T}\right)\right\}\right. \\
& \left.+\frac{1}{\sin ^{2} \theta} \frac{\partial}{\partial \phi}\left(D_{0} \frac{\partial}{\partial \phi}\right)\right] f
\end{aligned}
\] 

这表明,该函数

\[
\exp \left(H_{0} M \cos \theta / k T\right)
\] 

实际上是静态场中的平衡分布。如果进一步将极度缩小的条件

\[
\omega_{0} \tau_{c} \ll 1
\] 

满足,松弛时间\(\tau_{0}\) 和\(\tau_{1}\) 变得相等,扩散张量\(D\) 变得各向同性,而公式(35)仅仅成为浸入粘稠液体中的电偶极的旋转扩散方程,正如多年前德拜所证明的那样。\({ }^{13)}\) 

平均磁矩\(\langle M\rangle\) 的运动方程很容易从公式(34)中推导出来,其公式为

\[
\begin{aligned}
& \frac{d\left\langle M_{x}\right\rangle}{d t}=-\omega_{0}^{\prime}\left\langle M_{y}\right\rangle-\frac{1}{T_{2}}\left\langle M_{x}\right\rangle-\frac{H_{0}\left\langle M_{x} M_{z}\right\rangle}{2 T_{1} k T}, \\
& \frac{d\left\langle M_{y}\right\rangle}{d t}=\omega_{0}^{\prime}\left\langle M_{x}\right\rangle-\frac{1}{T_{2}}\left\langle M_{y}\right\rangle-\frac{H_{0}\left\langle M_{y} M_{z}\right\rangle}{2 T_{1} k T}, \\
& \frac{d\left\langle M_{z}\right\rangle}{d t}=-\frac{1}{T_{1}}\left\langle M_{z}\right\rangle+\frac{H_{0}\left\langle M_{x}^{2}+M_{y}^{2}\right\rangle}{2 T_{1} k T} .
\end{aligned}
\] 

由于基本方程(30)的非线性,这些方程包括较高磁矩的平均数,一般来说构成一个方程的层次结构。如果对平衡的偏离很小,这个层次结构可以被解耦。在平衡状态下,方程(36)的第三个方程给出了

\[
\left\langle M_{z}\right\rangle=H_{0}\left\langle M_{x}^{2}+M_{y}^{2}\right\rangle / 2 k T=H_{0}\left\langle M_{x}^{2}\right\rangle / k T
\] 

公式(37)表明,磁感应强度\(\chi\) 由以下公式给出

\[
\chi=\left\langle M_{x}^{2}\right\rangle_{0} / k T
\] 

符合统计力学的著名结果。

只要外场不是太强,以至于\(H_{0} M \ll k T\) ,(36)的第一和第二方程右边的最后一项作为阻尼对\(\left\langle M_{x}\right\rangle\) 和\(\left\langle M_{y}\right\rangle\) 可以忽略不计,因此我们有布洛赫方程

\[
\begin{aligned}
& \frac{d}{d t}\left\langle M_{x}\right\rangle=-\omega_{0}^{\prime}\left\langle M_{y}\right\rangle-\frac{1}{T_{2}}\left\langle M_{x}\right\rangle, \\
& \frac{d}{d t}\left\langle M_{y}\right\rangle=\omega_{0}^{\prime}\left\langle M_{x}\right\rangle-\frac{1}{T_{2}}\left\langle M_{y}\right\rangle, \\
& \frac{d}{d t}\left\langle M_{z}\right\rangle=-\frac{1}{T_{1}}\left(\left\langle M_{z}\right\rangle-\chi H_{0}\right) .
\end{aligned}
\] 

值得注意的是,尽管我们从公式(28)中的Landau-Lifshitz摩擦开始,但我们还是被引向了Bloch型松弛。

方程(34)包含平均磁矩向对应于瞬时磁场的平衡值的松弛。这个方程可用于研究当外场\(H_{0}(t)\) 在时间上变化时,磁旋的响应和噪声,只要它在时间上的变化与波动场相比足够缓慢。

\section{§4. Concluding remarks} 
我们已经表明,在代表与周围环境相互作用的随机场的影响下,可以为经典自旋制定一个现象学理论。如果假定随机场是弱的、适度的和充分短相关的,就有可能推导出自旋矩随机运动的福克-普朗克方程。由于有摩擦力伴随着随机场,福克-普朗克方程保证了在有限温度下接近平衡。从形式上看,将这种处理方法扩展到量子自旋似乎很容易。在量子力学中,我们不再能确定自旋的所有成分,因此我们不得不从密度矩阵的角度来考虑,其运动方程为

\[
\frac{\partial \rho}{\partial t}=\frac{1}{i \hbar}[\mathcal{H}(t), \rho]
\] 

自旋的密度矩阵可以用算子球面谐波展开,其前几个成员为

\[
\begin{aligned}
& 1, S_{x}, S_{y}, S_{z}, S_{x} S_{y}+S_{y} S_{x}, S_{y} S_{z}+S_{z} S_{y}, S_{z} S_{x}+S_{x} S_{z} \\
& S_{x}^{2}-S_{y}^{2}, 2 S_{z}^{2}-S_{x}^{2}-S_{y}^{2}, \cdots .
\end{aligned}
\] 

因此,我们可以把

\[
\rho=A+c \boldsymbol{M}(t) \boldsymbol{S}+\cdots,
\] 

其中膨胀系数\(\boldsymbol{M}\) 是\(\boldsymbol{S}\) 在\(t\) 时间的量子力学期望值,即

\[
\boldsymbol{M}(t)=\operatorname{Tr} \rho(t) \mathbf{S},
\] 

是归一化是这样选择的,即

\[
c^{-1}=\operatorname{Tr} S_{x}^{2}=(2 S+1) S(S+1) .
\] 

如果汉密尔顿\(\mathcal{H}(t)\) 是

\[
\mathscr{H}(t)=\left(\boldsymbol{H}_{0}+\boldsymbol{H}^{\prime}\right) \gamma \hbar \boldsymbol{S}
\] 

式(40)给出了方程

\[
\frac{d \boldsymbol{M}}{d t}=\gamma\left(\boldsymbol{H}_{0}+\boldsymbol{H}^{\prime}\right) \times \boldsymbol{M}
\] 

这是\(\boldsymbol{M}(t)\) 的随机运动方程,或\(\mathbf{S}\) 的量子力学期望。从现象学的角度来看,这个方程可以再次补充一个摩擦项,以保证接近热平衡。那么,在前面几节中讨论的处理方法可被视为适用于量子自旋和经典自旋。对\(\boldsymbol{M}\) 分布的平均意味着对随机场\(\boldsymbol{H}^{\prime}(t)\) 的随机过程集合的量子力学期望的平均。然而,关于物理观测的意义,仍然存在着某些模糊不清的地方。例如,如何解释\(\left\langle\boldsymbol{M}\left(t_{1}\right) \boldsymbol{M}\left(t_{2}\right)\right\rangle\) 型的相关函数并不十分清楚。一般来说,量子系统的布朗运动理论目前处于相当不令人满意的状态,仍然是统计力学中的一个突出问题。在此,我们将不再进一步探讨这个概念上的复杂问题。

这项工作的主要部分是在1963年夏天完成的,当时我们住在芝加哥大学的金属研究所。我们希望对S.A.Rice教授和M.H.Cohen教授表示感谢,感谢他们对这个夏天的愉快回忆。我们也很高兴将这篇小作品献给长宫武夫教授的60岁生日。


